\documentclass[../document.tex]{subfiles}

\begin{document}
\subsection{Глоссарий}
\begin{itemize}
    \item Веб-приложение - клиент-серверное приложение, в котором клиент взаимодействует с веб-сервером при помощи браузера. Логика веб-приложения распределена между сервером и клиентом, хранение данных осуществляется, преимущественно, на сервере, обмен информацией происходит по сети. [1]
    \item RPG – знаменитый жанр компьютерных и видео игр, где основой игрового процесса является отогревание определенной роли. Игрок берёт под контроль определённого героя или героиню, с набором стандартных навыков, характеристик и умений. [2]
    \item Геймификация — это внедрение игровых форм в неигровой контекст: работу, учебу и повседневную жизнь. [3]
\end{itemize}
\subsection{Описание предметной области}
\par Разработка веб-приложений – это перспективное направление, которое объединяет в себе доступность и эффективность. Такие программы не требуют от пользователя установки, из-за чего пользуются популярностью. В частности, в формате веб-приложений распространены трекеры задач. От подобной программы пользователь ожидает совместимости со всеми типа устройств, поэтому веб-приложение является оптимальным выбором для разработчика.
\subsection{Неформальная постановка задачи}
\par Создать веб-приложение для управления задачами, основанный на принципах геймификации, который будет мотивировать пользователей к выполнению задач путем использования игровых механик и элементов.
\pagebreak
\subsection{Обзор существующих методов решения}
\begin{flushright}
    Таблица 1
\end{flushright}
\par
\begin{tabular}{ | m{5em} | m{5em} | m{5em} | m{5em} | m{5em} | }
    \hline
    Продукт  & Игровые механики & Понятный интерфейс & Доступность                           & Платные функции                    \\
    \hline
    Todoist  & Отсутств.        & Присутств.         & Все платформы                         & Подписка на дополнительные функции \\
    \hline
    Weeek    & Отсутств.        & Присутств.         & Все платформы                         & Подписка на дополнительные функции \\
    \hline
    Things   & Отсутств.        & Присутств.         & только Mac и iOS                      & Полностью платное приложение       \\
    \hline
    Trello   & Отсутств.        & Присутств.         & веб-приложение                        & Отсутств.                          \\
    \hline
    Habitica & Присутств.       & Присутств.         & веб-приложение и мобильные устройства & Подписка на дополнительные функции \\
    \hline
\end{tabular}
\pagebreak
\subsection{План работ}
\begin{flushright}
    Таблица 2
\end{flushright}
\par
\begin{tabular}{ | l | l | }
    \hline
    Этап                                            & Месяц          \\
    \hline
    Исследование рынка, паттернов, полезных практик & Сентябрь       \\
    \hline
    Составление модели сервиса, функционала         & Сентябрь       \\
    \hline
    Составление схемы базы данных                   & Октябрь        \\
    \hline
    Настройка репозитория                           & Октябрь        \\
    \hline
    Дизайн интерфейса                               & Октябрь        \\
    \hline
    Создание скелета проекта                        & Октябрь        \\
    \hline
    Поднятие базы данных                            & Ноябрь         \\
    \hline
    Настройка маршрутов                             & Ноябрь         \\
    \hline
    Отрисовка спрайтов                              & Ноябрь-Декабрь \\
    \hline
    Вёрстка интерфейса                              & Ноябрь-Январь  \\
    \hline
    Реализация серверного функционала               & Декабрь-Январь \\
    \hline
    Внедрение клиентского функционала               & Январь         \\
    \hline
    Тестирование                                    & Февраль        \\
    \hline
    Доработка                                       & Февраль        \\
    \hline
    Тест и релиз продукта                           & Февраль        \\
    \hline
\end{tabular}
\end{document}
